% $Id: srm-vm.tex,v 1.62 2023/11/15 19:41:22 leavens Exp leavens $
\documentclass[11pt,letterpaper]{article}
\usepackage[T1]{fontenc}
\usepackage{times}
\usepackage{textcomp} %required to set upquote flag and for luximono
%\usepackage[scaled=0.81]{luximono}
\usepackage{enumitem}
\usepackage{listings}
\usepackage{graphics}
\usepackage{color}
\usepackage{ragged2e}
\lstset{language=C,basicstyle=\ttfamily,commentstyle=\rmfamily\itshape}
\lstset{morecomment=[l]{//}, %use comment-line-style!
        morecomment=[s]{/*}{*/}} %for multiline comments
\lstset{upquote=true} %FH
% The following is needed for C programs
\lstset{showspaces=false,showstringspaces=false}
%\lstset{mathescape=true}
\input{boxed-figure-new}
% getting files for figures
\newcommand{\LSTFILE}[1]{
\lstinputlisting[basicstyle=\small\ttfamily]{#1}
}
\newcommand{\LSTFILESLICE}[2]{
\lstinputlisting[basicstyle=\small\ttfamily,linerange={#1}]{#2}
}
%
\newcommand{\LSTCFILE}[1]{
\lstinputlisting[language=C,basicstyle=\small\ttfamily]{#1}
}
\newcommand{\LSTCFILESLICE}[2]{
\lstinputlisting[language=C,basicstyle=\small\ttfamily,linerange={#1}]{#2}
}
\usepackage[bookmarks=false]{hyperref}
%\usepackage{varioref}  % doesn't work with hyperref, it seems, sigh
%\renewcommand{\reftextfacebefore}{on the previous page}
%\renewcommand{\reftextfaceafter}{on the next page}
\newcommand{\tabref}[1]{Table~\ref{#1}}  % vref from varioref
\newcommand{\tabnref}[1]{Table~\ref{#1}}  % doesn't use varioref
\newcommand{\figref}[1]{Figure~\ref{#1}}  % vref from varioref
\newcommand{\fignref}[1]{Figure~\ref{#1}}  % doesn't use varioref
\newcommand{\secref}[1]{Section~\ref{#1}}  % vref from varioref
\newcommand{\secnref}[1]{Section~\ref{#1}}  % doesn't use varioref

\input{obey}
\input{grammar}
\renewcommand{\nonterm}[1]{\mbox{$\langle$\textrm{#1}$\rangle$}}
\renewcommand{\arbno}[1]{\{#1\}}
\newcommand{\gramquote}[1]{\textrm{`}#1\textrm{'}}
\input{use-full-height}
\input{use-full-width}
\input{continuedenum}

\setlength\RaggedRightParindent{\parindent}
% $Id: srm-isa.tex,v 1.3 2023/09/16 12:41:27 leavens Exp $
% Macros for the ISA follow
\newcommand{\WS}{32}
\newcommand{\WSP}{31}
\newcommand{\WSinBytes}{4}
\newcommand{\LOGofRegs}{5}
\newcommand{\WSminusOP}{26}
\newcommand{\INITIALSTACKBOTTOM}{4096}
\newcommand{\NAMEDREG}[1]{\textsf{\$#1}}
\newcommand{\BP}{\FP}
\newcommand{\GP}{\NAMEDREG{gp}}
\newcommand{\FP}{\NAMEDREG{fp}}
\newcommand{\SP}{\NAMEDREG{sp}}
\newcommand{\RA}{\NAMEDREG{ra}}
\newcommand{\RS}{\NAMEDREG{rs}}
\newcommand{\RT}{\NAMEDREG{rt}}
\newcommand{\RD}{\NAMEDREG{rd}}
\newcommand{\VZERO}{\NAMEDREG{v0}}
\newcommand{\AZERO}{\NAMEDREG{a0}}
\newcommand{\AONE}{\NAMEDREG{a1}}
\newcommand{\ATWO}{\NAMEDREG{a2}}
\newcommand{\GPR}[1]{\textrm{GPR[}#1\textrm{]}}
\newcommand{\IMMED}{\textit{i}}
\newcommand{\OFFSET}{\textit{o}}
\newcommand{\PC}{\textit{PC}}
\newcommand{\HI}{\textit{HI}}
\newcommand{\LO}{\textit{LO}}
\newcommand{\IR}{\textit{IR}}
\newcommand{\SIGNEXTENDNAME}{\textrm{sgnExt}}
\newcommand{\SIGNEXTEND}[1]{\SIGNEXTENDNAME(#1)}
\newcommand{\ZEROEXTENDNAME}{\textrm{zeroExt}}
\newcommand{\ZEROEXTEND}[1]{\ZEROEXTENDNAME(#1)}
\newcommand{\FORMOFFSETNAME}{\textrm{formOffset}}
\newcommand{\FORMOFFSET}[1]{\FORMOFFSETNAME(#1)}
\newcommand{\FORMADDRESSNAME}{\textrm{formAddress}}
\newcommand{\FORMADDRESS}[2]{\FORMADDRESSNAME(#1,#2)}
\newcommand{\MEMORYNAME}{\textrm{memory}}
\newcommand{\MEMORY}[1]{\ensuremath{\MEMORYNAME[#1]}}
\newcommand{\MOD}{\ensuremath{\:\hbox{\%}\:}}
\newcommand{\AND}{\ensuremath{\wedge}}
\newcommand{\OR}{\ensuremath{\vee}}
\newcommand{\XOR}{\ensuremath{\:\textrm{xor}\:}}
\newcommand{\LSHIFT}{\ensuremath{\:\texttt{<<}\:}}
\newcommand{\RSHIFT}{\ensuremath{\:\texttt{>>}\:}}
\newcommand{\GETS}{\ensuremath{\leftarrow}}

\newcommand{\IFO}[2]{\textrm{\textbf{if} \ensuremath{#1} \textbf{then} \ensuremath{#2} }}
\newcommand{\IF}[3]{\textrm{\textbf{if} #1 \textbf{then} #2 \textbf{else} #3}}
\newcommand{\EQUALS}{\ensuremath{=}} % or {\ensuremath{\:\texttt{==}\:}}
%
\bibliographystyle{plain}
%
\title{Simplified RISC Machine Manual \\
       (\lstinline!$Revision: 1.62 $!)}
\author{Gary T. Leavens \\
        Leavens@ucf.edu}
\begin{document}

\maketitle
\begin{abstract}
This document defines the machine code of the Simplified RISC Machine VM for
use in the Systems Software class (COP 3402) at UCF.
\end{abstract}

\section{Overview}

The Simplified RISC Machine (SRM) processor's instruction set architecture
(ISA) is simplified from the MIPS processor's ISA \cite{Kane-Heinrich92}. 
In particular, SRM is a little-endian machine with $\WS$-bit ($\WSinBytes$-byte) words.
All instructions are also $\WS$-bits wide and there is no floating-point support, kernel mode support, or other advanced features.

\subsection{Inputs and Outputs}

\subsubsection{Binary Object Files}

The VM is passed a single file name on its command line as an
argument.
Optionally the VM be passed a single command line option, either
\texttt{-p} or \texttt{-t}.

When given a \texttt{-p} argument followed by a binary object file
name, the VM loads the binary object file and prints the assembly
language form of the program, see \secref{sec:printing} details.

When given a \texttt{-t} argument followed by a binary object file
name, the VM runs the binary object file with tracing output enabled.
By default, no tracing happens, but users can turn on tracing under
program control by executing the \texttt{STRA} system call.

The remainder of this section is concerned
with what the VM does when it only is given a binary object file name
on its command line as an argument.

The file name given to the VM must be the name of a (readable)
binary object file containing the program that the VM should execute.
For example, if the VM's executable is named \texttt{vm} and the program it
should run is contained in the file named \texttt{test.bof}
(and both these files are in the current directory), then the VM should
execute the program in the file \texttt{test.bof} by executing
the following command in the Unix shell.

\begin{lstlisting}
  ./vm test.bof
\end{lstlisting}

The format of a binary object file (BOF) is given in the header file
\texttt{bof.h}, which is shown in part in \figref{fig:bofh}.
A BOF starts the header, then the instructions (also in
binary form) follow, followed by the initial values of data.
This layout of binary object files is shown in \figref{fig:boffilelayout}.

The header of a binary object file starts with a 4-character field
that contains the characters ``\lstinline!BOF!''
(and a null character); this kind of ``magic number'' is commonly used
to identify files in Unix.
The magic number is followed (in the BOF header) by the starting address of
the program's code and the length of the program's code (in bytes), which
constitutes the ``text'' section of the binary object file.
These are followed by the starting address of the data section and its
length (in bytes).
The data section contains the global/static variables that the program uses.
Finally, the header contains the initial value for the stack and frame
pointers, which is the address (in bytes) of the bottom of the runtime
stack.

\begin{figure}[htbp]
\LSTCFILESLICE{1-24,75-76}{bof.h}
\caption{The \texttt{bof.h} header file that defines the format and
  operations for binary object files.}
\label{fig:bofh}
\end{figure}

\begin{figure}[htbp]
\includegraphics*{BOF_layout.png}
\caption{The layout of a binary object file.}
\label{fig:boffilelayout}
\end{figure}

\subsubsection{Initial/Default Values}
\label{sec:initial}

The memory of the machine starts at all zero ($0$) values.
Then the instructions specified by the given binary object file
(as named on the command line) are loaded into memory,
starting at address $0$, making the contents of the first $N$ bytes
(where $N$ is divisible by $\WSinBytes$) 
of memory be the same as the $N$ bytes following the header itself in
the binary object file; here $N$ is the same as the header's value of
the \lstinline!text_length! field.
Following those $N$ bytes are the bytes of the data section.
These are loaded into the memory starting at the data start address
given in the header; thus any initial values are copied from the
data section of the binary object file into VM's memory.

When the program starts executing:
\begin{itemize}
\item
the register {\GP} is set to the start address of the data section
given in the header,
which must be divisible by $\WSinBytes$,

\item
the registers {\FP} and {\SP} are both is set to the stack bottom
address given in the header, which must be divisible by $\WSinBytes$
and strictly greater than the start address of the data section,

\item
to simulate a call of the program (by the OS) the {\AZERO} register is
set to the same value as {\FP}, and

\item
the program counter {\PC} is set to the text section's start address,
which must be divisible by $\WSinBytes$ and strictly less than the
data section's start address.
\end{itemize}

% \clearpage

\subsection{The Running Program's Input and Output}

When the program executes instructions to read or write characters,
these are read from standard input (\texttt{stdin}) and written to
standard output (\texttt{stdout}).

However, note that if you want the program to read a character, typing a single
character (say \texttt{x}) into the terminal (i.e., to the shell) 
while the program is running will not send that character immediately
to the program, as standard input is buffered by default.  To send
characters to the program it is best to use a pipe or file redirection
in the Unix shell; for example, to send the two characters \texttt{x}
and \texttt{y} (followed by a newline character)
to the VM running the program \texttt{progfile.bof} one
could use the following command at the Unix shell:

\begin{lstlisting}
  echo xy | ./vm progfile.bof
\end{lstlisting}

Another command that would accomplish the same thing is to put the
characters to be input into a file (using a text editor), say
\texttt{xy-input.txt} and then to use the following Unix command.

\begin{lstlisting}
  ./vm progfile.bof < xy-input.txt
\end{lstlisting}

\subsection{Tracing Output}

When used with the \texttt{-t} option or after the \texttt{STRA}
system call is executed,
the VM produces output that traces the its execution on \texttt{stdout}.
(This tracing output can be turned off under program control
by executing a \texttt{NOTR} system call instruction.)
(See \secref{sec:syscall} for more information about the \texttt{STRA}
and \texttt{NOTR} system call instructions.)

The tracing output shows the initial state of the VM, then for each
instruction executed, it shows the address (in bytes) of that
instruction and then the assembly language form of that instruction.

The state of the machine shown in tracing output shows:
the values in the PC, the values of the HI and LO registers (if those
are not zero) and the values in all the general purpose registers,
then the memory starting at the address in $\GPR{\GP}$
with locations containing zeros only indicated with ``\texttt{...}''.
and then the data between the addresses between $\GPR{\SP}$ and
the stack bottom address specified in the binary object file.
The idea is that those addresses should include the runtime stack.
When showing the memory, locations containing zeros are only
indicated with ``\texttt{...}''.
For example, when the binary object file that is assembled from the
assembly code shown in \figref{fig:test0asm} is executed,
which is found in the file \texttt{vm\_test0.bof},
it produces the output shown in \figref{fig:test0output}.

\begin{figure}
\LSTFILE{vm_test0.asm}
\caption{The SRM assembly language file \texttt{vm\_test0.asm}.} 
\label{fig:test0asm}
\end{figure}

\begin{figure}
\lstinputlisting[basicstyle=\ttfamily\scriptsize]{vm_test0.out}
\caption{The tracing output of running the VM (with the \texttt{-t} option)
  on the file \texttt{vm\_test0.bof}
  (which is the result of using the assembler on \texttt{vm\_test0.asm}),
  with the command \texttt{./vm vm\_test0.bof},
  as would be printed on standard output.} 
\label{fig:test0output}
\end{figure}

\subsection{Printing the Program}
\label{sec:printing}

When the VM is given the argument \texttt{-p} option followed by a
binary object file name, it first loads the instructions and data from the
given binary object file, then it prints what was loaded in assembly
language format, and then the VM exits (without running the program). This
can be helpful for understanding what a program is doing. This output
is shown in \figref{fig:test0listing}.

\begin{figure}
\LSTFILE{vm_test0.lst}
\caption{The output of running \texttt{./vm -p vm\_test0.bof}. 
  (where the file \texttt{vm\_test0.bof} is the result of assembling the file
  \texttt{vm\_test0.asm} that is shown in \figref{fig:test0asm}).} 
\label{fig:test0listing}
\end{figure}

\subsection{Error Outputs}

All error messages (e.g., for division by zero) are sent to
standard error output (\texttt{stderr}).

\subsection{Exit Code}

When the machine halts normally, it exits with a zero error code
(which indicates success on Unix).  However, when the machine
encounters an error it halts and exits with a non-zero exit code
(which indicates failure on Unix).

\section{Architecture}

In SRM, words are $\WS$ bits ($\WSinBytes$ bytes).
These bits can be interpreted as an integer or as an address.
The machine is byte addressed but instructions must always be at an
address that is on a word boundary (i.e., whose address is evenly
divisible by $\WSinBytes$).

\subsection{Registers}

\subsubsection{General Purpose Registers and Their Names}

The SRM is a register machine with 32
general purpose registers\footnote{What we call
``registers'' in this document are simply important concepts that
simulate what would be registers in a hardware implementation of the
virtual machine. For the VM program, these registers would be implemented
as variables.},
numbered 0 to 31 (inclusive).
These are all $\WS$-bit registers.
Since there are 32 registers,
instructions use $\LOGofRegs$ bits to specify them.

Register 0 cannot be written to, and when read its value is always 0.

Conventions (from the MIPS architecture \cite{Kane-Heinrich92}) are
followed for these registers and their names,
as shown in \tabref{tab:registers}.
The names shown in \tabnref{tab:registers} are conventional ones.

\begin{table}[htb]
  \caption{SRM Register Numbers, Use, and Names}
  \label{tab:registers}
\begin{center}
\begin{tabular}{| l | l | l |}
  \hline
  Number & Use & Name\\
  \hline
  $0$ & always 0 (can't write to this register!) & \\
  $1$ & assembler temporary & \NAMEDREG{at} \\
  $2-3$ & function results & \NAMEDREG{v0}, \NAMEDREG{v1} \\
  $4-7$ & function arguments & \NAMEDREG{a0}$-$\NAMEDREG{a3} \\
  $8-15$ & temporaries & \NAMEDREG{t0}$-$\NAMEDREG{t7} \\
  $16-23$ & temporaries & \NAMEDREG{s0}$-$\NAMEDREG{s7} \\
  $24-25$ & temporaries & \NAMEDREG{t8}, \NAMEDREG{t9} \\
  $26-27$ & reserved for use by OS (don't use!) & \\
  $28$ & globals pointer & $\GP$ \\
  $29$ & stack pointer & $\SP$ \\
  $30$ & frame pointer & $\FP$ \\
  $31$ & return address & $\RA$ \\
  \hline
\end{tabular}
\end{center}
\end{table}

\subsubsection{Special Purpose Registers}

SRM also has a few special registers. The registers are named:

\begin{itemize}
\item
  {\PC}, the program counter which holds the address of the next
  instruction to execute,

\item
  {\HI}, the high part (i.e., most significant bits) of the result of a multiplication or the
  remainder in a division,

\item
  {\LO}, the low part (i.e., least significant bits) of the result of
  a multiplication or the quotient in a division.
\end{itemize}

The $\PC$ register is manipulated by jump instructions.
The $\HI$ and $\LO$ registers are read by instructions that move their contents into another register.

\subsubsection{Calling Convention}

The calling convention on the SRM follows the calling convention on
the MIPS processor.

That is, the caller must save registers $1-15$, and $24-25$
if they will be needed after a call (and then restores them when
needed).

The callee saves (and restores before it returns) registers $16-23$
and $29-31$, if it uses (writes) them.

(Furthermore, register $0$ cannot be changed and registers $1$ and
$28$ should not be changed by a hand-written routine.
Registers $26-27$ should not be changed by user code.)

Note that the jump-and-link (\texttt{JAL}) instruction
does not save any registers except the $\PC$,
and it will save that in register $31$.

\subsection{Binary Instruction Format}

In object code,
all instructions are one word long and start with a 6-bit opcode.
However, instructions may have one of several formats, with the format
depending on the opcode (called ``op'' below).
The fields of each instruction format shown in \tabref{tab:instrfmts}
are followed by their width in bits;
for example the op field is 6 bits wide.

\begin{table}[htb]
  \caption{SRM Instruction Formats}
  \label{tab:instrfmts}
\begin{itemize}
\item Register/computational type instruction format: \\
\begin{tabular}{|c|p{6ex}{c}|p{6ex}{c}|p{6ex}{c}|p{10ex}|p{9ex}|}
  \hline
  op:6 & \multicolumn{1}{|c|}{rs:5} & \multicolumn{1}{|c|}{rt:5} & \multicolumn{1}{|c|}{rd:5} & \multicolumn{1}{|c|}{shift:5} & \multicolumn{1}{|c|}{func:6} \\
  \hline
\end{tabular}

\item System call instructions, whose format is a variant of the register type
instruction format, but with a func field value of 12: \\
\begin{tabular}{|c|p{6ex}{c}|p{6ex}{c}|p{6ex}{c}|p{10ex}|p{9ex}|}
  \hline
  op:6 & \multicolumn{4}{|c|}{~~~~~~~~~~~~~code:20~~~~~~~~~~~~~~~~} & \multicolumn{1}{|c|}{func:6} \\
  \hline
\end{tabular}

\item
Immediate operand type instruction format: \\
\begin{tabular}{| c | p{6ex}{c} | p{6ex}{c} | p{16ex} |}
  \hline
  op:6 & \multicolumn{1}{|c|}{rs:5} & \multicolumn{1}{|c|}{rt:5} & \multicolumn{1}{c|}{~~~~~~~~~immed:16~~~~~~~~~~~} \\
  \hline
\end{tabular}

\item
Jump type instruction format: \\
\begin{tabular}{| c | p{26ex}|}
  \hline
  op:6 & \multicolumn{1}{c|}{~~~~~~~~~~~~~~~~~~~~~addr:26~~~~~~~~~~~~~~~~~~~~~~~} \\
  \hline
\end{tabular}
\end{itemize}
\end{table}

The list of instructions and details on their execution appears in
Appendix \ref{appsec:A}.

\subsection{Machine Cycles}

The SRM instruction cycle conceptually does the following for each
instruction:

\begin{enumerate}
\item
  Let $\IR$ be the instruction at the location that {\PC} indicates.
  (Note that $\IR$ could be considered to be the contents of a register.)

\item
  The {\PC} is made to point to the next instruction
  (i.e., it is set to the address $\PC + \WSinBytes$).

\item
  Then the instruction in $\IR$ is executed.
  The op component of this instruction ($\IR$.op) indicates the operation
  to be executed.
  For example, if $\IR$.op encodes the instruction JR,
  then the machine jumps to the specified address by
  setting the $\PC$ register (to the contents of the given register).
\end{enumerate}

\subsection{Size Limits}

The following constant defines the size of the memory for the VM.

\LSTCFILESLICE{10-10}{machine.h}

You might need to copy this definition into your program.

Note that \lstinline!BYTES_PER_WORD! is defined to be 4 in
\lstinline!machine_types.h!, see \figref{fig:machinetypes}.

\subsection{Invariants}

The VM enforces the following invariants and will halt with an error
message (written to stderr) if one of them is violated:

\newcommand{\BPW}{\mbox{\texttt{BYTES\_PER\_WORD}}}
\begin{itemize}
\item
  $\PC \MOD \BPW = 0$,

\item
  $\GPR{\GP} \MOD \BPW = 0$,

\item
  $\GPR{\SP} \MOD \BPW = 0$,

\item
  $\GPR{\FP} \MOD \BPW = 0$,

\item
  $0 \leq \GPR{\GP}$,

\item
  $\GPR{\GP} < \GPR{\SP}$,

\item
  $\GPR{\SP} \leq \GPR{\FP}$,

\item
  $\GPR{\FP} < \mbox{\texttt{MEMORY\_SIZE\_IN\_BYTES}}$,
   
\item
  $0 \leq \PC$, 

\item
  $\PC < \mbox{\texttt{MEMORY\_SIZE\_IN\_BYTES}}$, and

\item
  $\GPR{0} = 0$.
\end{itemize}

\appendix
\section{Appendix A}
\label{appsec:A}

In the following tables, italicized names (such as $s$) are
meta-variables that refer to integers.
If an instruction's field is
notated as $-$, then its value does not matter (we use $0$ as a
placeholder for such values in examples).

All numbers appearing in the following table are in decimal (base 10) notation.

\subsection{Register/Computational Instructions}

Note that all of the instructions in table \tabref{tab:rformatinstrs}
have an opcode of $0$,
and a function specified by the func field. They each also have 3 register
arguments: rs, rt, and rd.
The contents of the general purpose register $r$ is
notated as $\GPR{r}$ in the table.
All numbers in the table are in decimal notation.

All arithmetic and logical operations are performed as for C
\textbf{int} values.
However, the right shift works on the contents of the register
$\GPR{t}$ in a logical manner, as if it were an \textbf{unsigned int},
so it should shift in zeros from the left.


\begin{table}[htbp]
\caption{Register Format Instructions}
\label{tab:rformatinstrs}
\begin{tabular}{|l|l|c|c|c|c|l|p{9cm}|}
\hline
~       & ~  & ~  & ~  & ~  & ~     &      & Comment \\
Name    & op & rs & rt & rd & shift & func & (Explanation) \\
\hline
ADD     & 0  &$s$ &$t$ &$d$  & -    & 33 & Add: $\GPR{d} \GETS \GPR{s} + \GPR{t}$ \\
\hline
SUB     & 0  &$s$ &$t$ &$d$  & -    & 35 & Subtract: $\GPR{d} \GETS \GPR{s} - \GPR{t}$ \\
\hline
MUL     & 0  &$s$ &$t$ & -   & -    & 25 & Multiply: Multiply
$\GPR{s}$ and $\GPR{t}$, \\
& & & & & & & putting the least significant bits in $\LO$ \\
& & & & & & & and the most significant bits in $\HI$. \\
& & & & & & & $(\HI,\LO) \GETS \GPR{s} \times \GPR{t}$ \\
\hline
DIV     & 0  &$s$ &$t$ & -   & -    & 27 & Divide (remainder in $\HI$, quotient in $\LO$): \\
        &    &    &    &     &      &    & $\HI \GETS \GPR{s} \MOD \GPR{t}$ \\
        &    &    &    &     &      &    & $\LO \GETS \GPR{s} / \GPR{t})$ \\
\hline
MFHI    & 0  & -  & -  & $d$ & -    & 16 & Move from HI: $\GPR{d} \GETS \HI$ \\
\hline
MFLO    & 0  & -  & -  & $d$ & -    & 18 & Move from LO: $\GPR{d} \GETS \LO$ \\
\hline
AND     &  0  &$s$ &$t$ & $d$ & -    & 36 & Bitwise And: $\GPR{d} \GETS \GPR{s} \AND \GPR{t}$ \\
\hline
BOR     &  0  &$s$ &$t$ & $d$ & -    & 37 & Bitwise Or: $\GPR{d} \GETS \GPR{s} \OR \GPR{t}$ \\
\hline
NOR     &  0  &$s$ &$t$ & $d$ & -    & 39 & Bitwise Not-Or: $\GPR{d} \GETS \neg(\GPR{s} \OR \GPR{t})$ \\
\hline
XOR     &  0  &$s$ &$t$ & $d$ & -    & 38 & Bitwise Exclusive-Or: $\GPR{d} \GETS \GPR{s} \XOR \GPR{t}$ \\
\hline
SLL     &  0  & - &$t$ & $d$ & $h$  & 0 & Shift Left Logical: $\GPR{d} \GETS \GPR{t} \LSHIFT h$ \\
\hline
SRL     &  0  & - &$t$ & $d$ & $h$  & 3 & Shift Right Logical: $\GPR{d} \GETS \GPR{t} \RSHIFT h$ \\
\hline
JR      &  0  &$s$ & 0 &  0  &  0   & 8 & Jump Register: $\PC \GETS \GPR{s}$ \\
\hline
SYSCALL &  0  & -  & - & -   & -   & 12 & System Call: (see \tabref{tab:syscalls}) \\
\hline
\end{tabular}
\end{table}

\newpage

\subsection{Immediate Type Instructions}

The instructions in \tabref{tab:immedinstrs} may have up to 2 register
arguments, and all have an immediate operand, which is a 16 bit value.

For arithmetic operations, the immediate value is sign-extended (to an
\textbf{int} value),
which is written in the explanations using the function ``{\SIGNEXTENDNAME}.'' 
For example, suppose $i$ is $-1$, which is FFFF in hexadecimal
notation;
then $\SIGNEXTEND{i}$ is FFFFFFFF in hexadecimal, which still
represents $-1$.

However, for logical operations, the immediate value is zero-extended,
which is written in the explanations using the function ``{\ZEROEXTENDNAME}.''
For example, suppose $i$ is $-1$, which is FFFF in hexadecimal
notation;
then $\ZEROEXTEND{i}$ is 0000FFFF in hexadecimal notation.

For the branches, the immediate value, $o$ is first shifted left 2
bits (multiplied by 4) and then sign-extended, which is written as
``{\FORMOFFSETNAME}'' in the table.
(Thus $\FORMOFFSET{o} = \SIGNEXTEND{4 \times o}$.)
Note that the resulting address is added to the address of the
instruction following the currently executing instruction, not the
address of the instruction itself, since the $\PC$ has already been advanced.
(As a simplification, the opcode for the BLTZ instruction is different
from that found in the MIPS architecture \cite{Kane-Heinrich92}.)

For loads and stores, $\MEMORY{a}$ denotes the contents of the
machine's memory at the byte address $a$.

\begin{table}[htbp]
\caption{Immediate format instructions:}
\label{tab:immedinstrs}  
\begin{tabular}{|l|r|c|c|c|p{11cm}|}
\hline
~    &      & ~  & ~  & ~     & Comment \\
Name & op   & rs & rt & immed & (Explanation) \\
\hline
ADDI  & 9   &$s$ &$t$ & $i$ & Add immediate: $\GPR{t} \GETS \GPR{s} + \SIGNEXTEND{i}$ \\
\hline
ANDI & 12    &$s$ &$t$ & $i$ & Bitwise And immediate: $\GPR{t} \GETS \GPR{s} \AND \ZEROEXTEND{i}$ \\
\hline
BORI & 13    &$s$ &$t$ & $i$ & Bitwise Or immediate: $\GPR{t} \GETS \GPR{s} \OR \ZEROEXTEND{i}$ \\
\hline
XORI & 14    &$s$ &$t$ & $i$ & Bitwise Xor immediate: $\GPR{t} \GETS \GPR{s} \XOR \ZEROEXTEND{i}$ \\
\hline
BEQ & 4     &$s$ &$t$ & $o$ & Branch on Equal: $\IFO{\GPR{s} = \GPR{t}}{\PC \GETS \PC + \FORMOFFSET{o}}$ \\
\hline
BGEZ & 1    &$s$ & 1  & $o$ & Branch $\geq$ 0: $\IFO{\GPR{s} \geq 0}{\PC \GETS \PC + \FORMOFFSET{o}}$ \\
\hline
BGTZ & 7    &$s$ & 0  & $o$ & Branch $>$ 0: $\IFO{\GPR{s} > 0}{\PC \GETS \PC + \FORMOFFSET{o}}$ \\
\hline
BLEZ & 6    &$s$ & 0  & $o$ & Branch $\leq$ 0: $\IFO{\GPR{s} \leq 0}{\PC \GETS \PC + \FORMOFFSET{o}}$ \\
\hline
BLTZ & 8    &$s$ & 0  & $o$ & Branch $<$ 0: $\IFO{\GPR{s} < 0}{\PC \GETS \PC + \FORMOFFSET{o}}$ \\
\hline
BNE  & 5    &$s$ &$t$ & $o$ & Branch Not Equal: $\IFO{\GPR{s} \neq \GPR{t}}{\PC \GETS \PC + \FORMOFFSET{o}}$ \\
\hline
LBU  & 36   &$b$ &$t$ & $o$ & Load Byte Unsigned: \\
     &      &    &    &     & $\GPR{t} \GETS \ZEROEXTEND{\MEMORY{\GPR{b} + \FORMOFFSET{o}}}$ \\
\hline
LW   & 35   &$b$ &$t$ & $o$ & Load Word ($\WSinBytes$ bytes): \\
     &      &    &    &     & $\GPR{t} \GETS \MEMORY{\GPR{b} + \FORMOFFSET{o}}$ \\
\hline
SB   & 40   &$b$ &$t$ & $o$ & Store Byte (least significant byte of $\GPR{t}$): \\
     &      &    &    &     & $\MEMORY{\GPR{b} + \FORMOFFSET{o}} \GETS \GPR{t}$ \\
\hline
SW   & 43   &$b$ &$t$ & $o$ & Store Word ($\WSinBytes$ bytes): \\
     &      &    &    &     & $\MEMORY{\GPR{b} + \FORMOFFSET{o}} \GETS \GPR{t}$ \\
\hline
\end{tabular}
\end{table}

\subsection{Jump Type Instructions}

The instructions in \tabref{tab:jumpinstrs} have a $\WSminusOP$-bit field ``addr''
which is used to form the address to jump to.
Forming this address is done by left-shifting the given ``addr''
field, $a$, by 2 bits,
and then concatenating the (4) high bits of the $\PC$ with those
$\WSminusOP+2$ bits to form a $\WS$-bit address.
This is written ``$\FORMADDRESS{\PC,a}$'' in the table.
For example if $a$ is DECADE in hexadecimal notation, and $\PC$ is
FFFACADE in hexadecimal notation, then $\FORMADDRESS{\PC}{a}$ is F37B2B78
in hexadecimal notation. (Note: if the high-order 4 bits of $\PC$
are 0, then $\FORMADDRESS{PC}{a}$ is equivalent to left-shifting $a$
by 2 bits.)

\begin{table}[htbp]
\caption{Jump Type Instructions}
\label{tab:jumpinstrs}  
\begin{tabular}{|r|l|l|p{11cm}|}
\hline
~      & ~  & ~    & Comment \\
Name   & op & addr & (Explanation) \\
\hline
JMP    & 2  & a    & Jump: $\PC \GETS \FORMADDRESS{PC}{a}$ \\
\hline
JAL    & 3  & a    & Jump and Link: $\GPR{\RA} \GETS \PC; \PC \GETS \FORMADDRESS{\PC}{a}$ \\
\hline
\end{tabular}
\end{table}

The Jump and Link (JAL) instruction does a subroutine call. It does
not explicitly manipulate the runtime stack.
% [[[Put more in about calling conventions here.]]]

\subsection{System Calls}
\label{sec:syscall}

System calls are used to provide operating system services.
System calls are made by instructions with op 0 and func 12 having the
following format (with code field made of the 20 bits of what would
be the rs, rt, rd, and shift fields of a register type instruction,
all combined).
The code field is used to specify the service requested.

System calls include exiting a program and various kinds of printing
and reading of character data (bytes).
These are described in \tabref{tab:syscalls},
using C library equivalents.
In the table, an entry of $-$ means that the contents of argument registers 
is not specified.
Otherwise, the contents of particular argument registers are used to
pass actual arguments to the system calls (the program must load the
actual argument values into those registers before making the call).
(Recall that the correspondence between named registers and register
numbers is given in \tabref{tab:registers}.)
All printing done by these instructions
is to the VM's standard output (stdout) and
reading is from the VM's standard input (stdin).

\begin{table}[htbp]
\caption{System Calls}
\label{tab:syscalls}
\begin{tabular}{|l|l|l|l|}
\hline
code & name  & arg. reg. & Effect (in terms of C std. library) \\
\hline
10   & EXIT  & - & \texttt{exit(0) // halt} \\
\hline
 4   & PSTR  & \AZERO & $\GPR{\VZERO} \GETS $ \texttt{printf("\%s",\&$\MEMORY{\GPR{\AZERO}}$)} \\
\hline
 5   & PINT  & \AZERO & $\GPR{\VZERO} \GETS $ \texttt{printf("\%d",\GPR{\AZERO})} \\
\hline
11   & PCH   & \AZERO & $\GPR{\VZERO} \GETS $\texttt{fputc($\GPR{\AZERO}$,stdout)} \\
\hline
12   & RCH   & -   & $\GPR{\VZERO} \GETS $ \texttt{getc(stdin)} \\
\hline
256   & STRA & -  & \textrm{start VM tracing; start tracing output} \\
\hline
257   & NOTR & -  & \textrm{no VM tracing; stop the tracing output} \\
\hline
\end{tabular}
\end{table}

In the PSTR instruction, the C standard library function
\texttt{printf} will expect a C pointer to characters as its
argument; this should be the address of those character's
representations in the memory starting at the VM address given by the
contents of $\GPR{\AZERO}$.

\section{Appendix B: Hints}

\subsection{Overall Structure of the Code}

To implement the SRM VM, the first thing to do is to decide on some
data structures to represent the VM's state: especially the memory and
registers.  You may want to represent the VM's memory using a
definition like that in \figref{fig:memoryrep}.
\begin{figure}[hbtp]
\begin{lstlisting}
#include "machine_types.h"
#include "instruction.h"
// a size for the memory (2^16 bytes = 64K)
#define MEMORY_SIZE_IN_BYTES (65536 - BYTES_PER_WORD)
#define MEMORY_SIZE_IN_WORDS (MEMORY_SIZE_IN_BYTES / BYTES_PER_WORD)

static union mem_u {
     byte_type bytes[MEMORY_SIZE_IN_BYTES];
     word_type words[MEMORY_SIZE_IN_WORDS];
     bin_instr_t instrs[MEMORY_SIZE_IN_WORDS];
} memory;
\end{lstlisting}
\caption{A possible way to represent the memory of the VM, which
  allows access to the same storage as bytes, words, or binary instructions.}
\label{fig:memoryrep}
\end{figure}
With this definition,
the VM's memory is represented as 3 arrays that share the same
storage: \lstinline!memory.bytes!, \lstinline!memory.words!, and
\lstinline!memory.instrs!.  For example,
the 4 bytes at byte address 36 can be accessed as
\lstinline!memory.bytes[36]!
or as \lstinline!memory.words[9]! or as \lstinline!memory.instrs[9]!.
This union definition allows VM's code to decide what view it wants
of the storage at each point in the implementation,
and whatever is changed in that view is
seen by all the other views.
Using a union in this way avoids lots of casting and bit manipulation.
Note that the C compiler considers \lstinline!memory.bytes! to have
the type \lstinline!byte_type[]!,
and \lstinline!memory.words! to have the type \lstinline!word_type[]!,
and \lstinline!memory.instrs! to have the type \lstinline!bin_instr_t[]!.

The registers can be \lstinline!word_type! variables
(where \lstinline!word_type! is defined in
\lstinline!machine_types.h!),
or an array of them.

Once the representation for memory and the registers is settled,
implement the loading process and get the \texttt{-p} option to work.
There is code in the disassembler that can be used as a model of what
to do.

To read from the binary object files (BOFs) you should use the
functions in the bof module.
In particular, use \lstinline!bof_read_open! to open such a file,
use \lstinline!bof_read_header! to read and return the header from a
BOF, then read each of the instructions using
\lstinline!instruction_read! (from the instruction module),
and \lstinline!bof_read_word! to read words in the data section of
the BOF.

You will find the function \lstinline!instruction_assembly_form!
from the instruction module
helpful in doing the printing of instructions. See the code in the
disassembler (\texttt{disasm.c} in particular) for a model of how to
use it.

After getting the \texttt{-p} option to work,
you need to implement the basic fetch-execute cycle for the VM
(without the \texttt{-p} option):
make a function that executes a single instruction and handles the tracing
and call the function to execute each instruction in a loop.

To implement the function that executes a single instruction,
have that function decide between the possible instructions
(using the functions from the instruction module) and
in each case carry out the effect of the instruction as described in
the ``Simplified RISC Machine Manual'', which is available on
Webcourses.
You can see the provided code in the
\lstinline!instruction_assembly_form!
function in the instruction module as an example of how to
decide what an instruction is.

\subsection{Writing Your Own Tests}

It is often helpful to write your own tests to execute just one or two
instructions during testing of the VM.
However, be sure to include an \texttt{EXIT} instruction in your test to
stop your program's execution!

To write your own tests you can use the provided assembler,
since the VM only takes binary object files as inputs.

The SRM assembler can be compiled using the provided Makefile and the
following command.
\begin{lstlisting}
make asm
\end{lstlisting}

We also provide documentation for the
assembly language in the course files on webcourses; see the file
named ``\texttt{srm-asm.pdf}''.

\subsection{Disassembler}

We also provide a disassembler, that does something similar to running the
virtual machine with the \texttt{-p} option; this program can be built
using the Makefile by issuing the command \texttt{make disasm}.
The way that the disassembler does its output, using the
\texttt{instruction} module, can be helpful in writing the code for
the \texttt{-p} option of the VM.

\subsection{Provided Files}

We provide all the source files used to build the assembler and the
disassembler. Many of these can be helpful in writing your VM
implementation. The following describes some of these.

\subsubsection{Makefile}

The provided \texttt{Makefile} describes how to compile and link programs
and run tests. This \texttt{Makefile} tells the GNU program
\texttt{make} \cite{GNUMake}
about dependencies between files that \texttt{make} uses
to decide when targets need to be built.

You should edit the Makefile's definition of \texttt{VM\_OBJECTS};
change that to be a complete list of the \texttt{.o} files
that are needed to build your virtual machine.
The list present in the Makefile for \texttt{VM\_OBJECTS} 
is what the course staff used,
but you might, for example, combine the main function,
which was in our files \texttt{machine\_main.c} and
\texttt{machine.c} (none of these are provided to you)
into a single file named \texttt{vm.c}, in which case would replace
our \texttt{machine\_main.o} and \texttt{machine.o} with your \texttt{vm.o}
On the other hand, you will need to leave the object file names in the
list that your code uses, such as \texttt{bof.o} and \texttt{utilities.o}.
If you receive an error message from the Unix linker/loader
(\texttt{ld}) about an ``undefined reference'' to a function or a
piece of data, then the solution is likely to include the relevant
object file in the list of \texttt{VM\_OBJECTS}.

You should not need to edit anything in the bottom half of the
Makefile, which is the ``developer's section'' used by the course staff.

There are several useful targets in the Makefile that can be used with
the \texttt{make} command. A \emph{target} is a name given on the
command line to \texttt{make}; for example \texttt{vm} is a target,
and running the command \texttt{make vm} should compile and link the
code needed to build your VM and create an executable program in the
file \texttt{vm} (or \texttt{vm.exe} on Windows).  The following is a
list of the targets in the Makefile that may be useful.

\newcommand{\bft}[1]{\textbf{\texttt{#1}}}
\begin{description}
\item{\bft{file.o}} compiles \texttt{file.c} if it (or \texttt{file.h}) is
  newer than \texttt{file.o}, producing a new copy of \texttt{file.o}.
  This works for any file name not just ``\texttt{file}'', as the
  Makefile has a general rule to compile \texttt{.c} files into
  \texttt{.o} files.

\item{\bft{vm}} compiles (if necessary) all the .o files named in the macro
  \texttt{VM\_OBJECTS} and links them together into an executable
  named \texttt{vm} (or \texttt{vm.exe} on Windows).

\item{\bft{vm\_test1.myo}} runs your virtual machine program (\texttt{./vm})
  on the input \texttt{vm\_test1.bof} and sends the (standard) output
  (and standard error output) to \texttt{vm\_test1.myo}.
  This works for any test file, not just ``\texttt{vm\_test1.bof}'' as
  the Makefile has a general rule for this.

\item{\bft{vm\_test1.myp}} runs your virtual machine program (\texttt{./vm})
  with the \texttt{-p} option on the input \texttt{vm\_test1.bof}
  and sends the (standard) output
  (and standard error output) to \texttt{vm\_test1.myp}.
  This works for any test file, not just ``\texttt{vm\_test1.bof}'' as
  the Makefile has a general rule for this.

\item{\bft{check-vm-outputs}} runs all of the provided tests in the
  \texttt{.bof} files and produces the corresponding \texttt{.myo}
  files using your VM (in \texttt{./vm}), and compares each one to the
  expected output in the corresponding \texttt{.out} file using the
  \texttt{diff} command. Each such test passes if no differences are detected.

\item{\bft{check-lst-outputs}} runs all of the provided tests in the
  \texttt{.bof} files and produces the corresponding \texttt{.myp}
  files using your VM (as \texttt{./vm -p}), and compares each one to the
  expected output in the corresponding \texttt{.lst} file using the
  \texttt{diff} command. Each such test passes if no differences are detected.

\item{\bft{check-outputs}} runs all of the provided tracing and listing
  tests (using the targets
  \texttt{check-lst-outputs} and \texttt{check-vm-outputs}).

\item{\bft{submission.zip}} runs all of the provided tests (using the
  \texttt{check-outputs} target) and creates a zip file that can be
  submitted for the assignment including your code and outputs from
  the tests.

\item{\bft{clean}} removes all the compiled object files (\texttt{*.o}) and testing
  output files (\texttt{*.myo} and \texttt{*.myp}) as well as the
  executable VM (files named \texttt{vm} and \texttt{vm.exe}) and the
  submission zip file (\texttt{submission.zip}). This allows you to
  start over from scratch, forcing \texttt{make} to build the do the
  work specified for the targets given, instead of thinking that they
  are up to date. (This is especially useful if some dependencies are
  not captured in the Makefile.)
\end{description}

\subsubsection{C Typedefs for SRM Machine Types}

We provide a module, \lstinline!machine_types! (the header
\lstinline!machine_types.h! is shown in \figref{fig:machinetypes}),
which defines some C equivalents for important types of data in the SRM.

\begin{figure}
\LSTCFILE{machine_types.h}
\caption{The header file of the \texttt{machine\_types} module, which
  provides some basic definitions for the VM.}
\label{fig:machinetypes}
\end{figure}

\subsubsection{Other modules provided}

The following gives a brief summary of the other provided code
modules, each of which consists of a \texttt{.c} file and a
\texttt{.h} file.

\begin{itemize}
\item
  \lstinline!bof!, which describes binary object files,

\item
  \lstinline!file_location!, which groups file names and line numbers,

\item
  \lstinline!instruction!, which describes machine instructions and
  provides several useful utilities for creating and printing
  instructions,

\item
  \lstinline!regname!, which provides access to the symbolic names of
  the SRM's general purpose registers,
  
\item
  \lstinline!utilities!, which describes several functions for error
  output, including \lstinline!bail_with_error!.
\end{itemize}

In addition, we provide code to build the assembler (including many of
the above and also the files
\lstinline!asm_main.c!, \lstinline!asm.y!, \lstinline!asm_lexer.l!,
\lstinline!asm_unparser.[ch]!, \lstinline!ast.[ch]!,
\lstinline!lexer.[ch]!, \lstinline!pass1.[ch]!,
\lstinline!assemble.[ch]!, and \lstinline!id_attrs!)
and the disassembler (including many mentioned above and also
\lstinline!disasm_main.c!, \lstinline!disasm.[ch]!).

You can use any of these provided files in your solution.

\bibliography{srm-vm.bib}

\end{document}
